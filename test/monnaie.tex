\documentclass[tikz,10pt,a4paper,onecolumn]{article}
\usepackage[francais]{babel}
\usepackage[margin=2cm]{geometry}
\usepackage[T1]{fontenc}
\usepackage{amsmath}
\usepackage{amsfonts}
\usepackage{amssymb}
\usepackage{listings}
\usepackage{graphicx}
\usepackage{tikz}

\lstset{
  language=Java,
  showstringspaces=false,
  formfeed=newpage,
  frame=single,
  frameround=tttt
  tabsize=2,
  commentstyle=itshape,
  morekeywords={models, lambda, forms}
}

\author{Fabien PETITJEAN}
\title{Rendre la monnaie de sa pièce}

\begin{document}
\maketitle

\begin{abstract}
Etude d'algorithmes \tikz \fill[orange] (1ex,1ex) circle (1ex); capables de faire l'appoint avec le moins de pièces possibles.
\end{abstract}

\section{L'algorithme glouton}
Ce qui vient d'abord à l'esprit, est l'algorithme que tout le monde utilise dans la vie quotidienne, souvent sans le savoir. Il consiste à donner en premier les plus grosses pièces dont on dispose et de continuer ainsi jusqu'à atteindre la somme voulue.

Voici des exemples d'appoint avec différents jeux de monnaie.

{{greedy [57, 10,7,5,1]}}
{{greedy [13, 5,2,1]}}
{{greedy [8, 5,4,1]}}

Dans ce dernier exemple, on voit qu'il aurait été plus intéressant d'utiliser deux pièces de 4 au lieu de 5, 1, 1 et 1. On se rend donc compte que cet algorithme ne donne pas toujours la solution optimale, c'est-à-dire celle qui minimise le nombre de pièces utilisées.

Mais il a quand même le grand avantage d'être simple à implémenter et à utiliser.

%\lstinputlisting{code.greedy.js}
\begin{lstlisting}
function money( total, coins ) {
   var result = [];
   var index = 0;
   while( total > 0 ) {
      while( index < coins.length && coins[index] > total ) index++;
      if( index >= coins.length ) break;
      result.push( coins[index] );
      total -= coins[index];        
   }
   return result;
}
\end{lstlisting}

\begin{itemize}
\item total : {\it nombre} - Somme totale à régler.
\item coins : {\it tableau} - Tableau des valeurs des différentes pièces, rangées par ordre décroissant.
\end{itemize}
Le retour est un tableau de pièces utilisées.

Toutes les fonctions présentées dans ce document ont la même signature.



\section{L'arbre des possibles}


\end{document}
