\documentclass[tikz,10pt,a4paper,onecolumn]{article}
\usepackage[francais]{babel}
\usepackage[margin=2cm]{geometry}
\usepackage[T1]{fontenc}
\usepackage{amsmath}
\usepackage{amsfonts}
\usepackage{amssymb}
\usepackage{listings}
\usepackage{graphicx}
\usepackage{tikz}

\lstset{
  language=Java,
  showstringspaces=false,
  formfeed=newpage,
  frame=single,
  frameround=tttt
  tabsize=2,
  commentstyle=itshape,
  morekeywords={models, lambda, forms}
}

\author{Fabien PETITJEAN}
\title{Rendre la monnaie de sa pièce}

\begin{document}
\maketitle

\begin{abstract}
Etude d'algorithmes \tikz \fill[orange] (1ex,1ex) circle (1ex); capables de faire l'appoint avec le moins de pièces possibles.
\end{abstract}

\section{L'algorithme glouton}
Ce qui vient d'abord à l'esprit, est l'algorithme que tout le monde utilise dans la vie quotidienne, souvent sans le savoir. Il consiste à donner en premier les plus grosses pièces dont on dispose et de continuer ainsi jusqu'à atteindre la somme voulue.

Par exemple, si on dispose de pièces de 5, 2 et 1 et que l'on doit payer 13€, on fera ceci :

{{greedy [57, 10,5,4,1]}}
{{greedy [13, 5,2,1]}}
{{greedy [8, 5,4,1]}}





\end{document}
