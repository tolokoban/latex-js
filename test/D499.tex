\documentclass[10pt,a4paper,onecolumn]{article}
\usepackage[utf8]{inputenc}
\usepackage[francais]{babel}
\usepackage[margin=2cm]{geometry}
\usepackage[T1]{fontenc}
\usepackage{amsmath}
\usepackage{amsfonts}
\usepackage{amssymb}
\usepackage{listings}
\usepackage{graphicx}
\usepackage{pstricks}

\lstset{
  language=Java,
  showstringspaces=false,
  formfeed=newpage,
  frame=single,
  frameround=tttt
  tabsize=2,
  commentstyle=itshape,
  morekeywords={models, lambda, forms}
}

\author{Fabien PETITJEAN}
\title{D499. Les carrés séquençables}

\begin{document}
\maketitle
\begin{abstract}
{\it Ce problème, proposé par Michel Lafond, provient du site {\tt http://diophante.fr}.}

\vspace{.5cm}

Le carré $C_n$ de côté $n$ est dit séquençable si on peut le paver entièrement et sans chevauchement avec les rectangles $R_1,R_2, \ldots,R_k$  dont les dimensions $[a_1,a_2], [a_3,a_4],\ldots,[a_{2k-1} ,a_{2k}]$   sont à l’ordre près les entiers  $1,2,3,\ldots,2k$.

Ci-après, à titre d'exemple, le carré séquençable $C_{13}$. Il est pavable avec les 5 rectangles $[1,2],[3,8],[4,5],[6,10]$ et $[7,9]$ dont les dimensions sont 1, 2, 3, 4, 5, 6, 7, 8, 9, 10.

\begin{pspicture}(0,0)(8,8)
  \psline[linewidth=.1]{<-}(2,1)
\end{pspicture}

PROUT.
\end{abstract}

\section{Solution}
Recherchons d'abord le nombre minimal de rectangles nécessaires à paver un carré dans les conditions du problème.



%\lstinputlisting{perms.js}
\end{document}
